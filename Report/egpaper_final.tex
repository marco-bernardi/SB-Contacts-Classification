\documentclass[10pt,twocolumn,letterpaper]{article}

\usepackage{cvpr}
\usepackage{times}
\usepackage{epsfig}
\usepackage{graphicx}
\usepackage{amsmath}
\usepackage{amssymb}
\usepackage{listings}
\usepackage{xcolor}
\usepackage[table,xcdraw]{xcolor}
\usepackage{caption}
\usepackage{makecell}
\usepackage[T1]{fontenc}
\usepackage{placeins}
\usepackage{booktabs}

\definecolor{lightergray}{rgb}{0.9,0.9,0.9}
\renewcommand{\arraystretch}{1.15}
\captionsetup[table]{skip=5pt}

\lstset{
    language=Python,
    basicstyle=\ttfamily\small,
    keywordstyle=\color{blue},
    stringstyle=\color{red},
    commentstyle=\color{green},
    showstringspaces=false,
    frame=single,
    breaklines=true
}
% Include other packages here, before hyperref.

% If you comment hyperref and then uncomment it, you should delete
% egpaper.aux before re-running latex.  (Or just hit 'q' on the first latex
% run, let it finish, and you should be clear).
\usepackage[breaklinks=true,bookmarks=false]{hyperref}

\cvprfinalcopy % *** Uncomment this line for the final submission

\def\cvprPaperID{****} % *** Enter the CVPR Paper ID here
\def\httilde{\mbox{\tt\raisebox{-.5ex}{\symbol{126}}}}

% Pages are numbered in submission mode, and unnumbered in camera-ready
%\ifcvprfinal\pagestyle{empty}\fi
\setcounter{page}{1}
\begin{document}

%%%%%%%%% TITLE
\title{Contacts Classification}

\author{Andrea Auletta\\
{\tt\small andrea.auletta@studenti.unipd.it}
% For a paper whose authors are all at the same institution,
% omit the following lines up until the closing ``}''.
% Additional authors and addresses can be added with ``\and'',
% just like the second author.
% To save space, use either the email address or home page, not both
\and
Marco Bernardi\\
{\tt\small marco.bernardi.11@studenti.unipd.it}
\and
Niccolò Zenaro\\
{\tt\small niccolò.zenaro@studenti.unipd.it}
}

\maketitle
%\thispagestyle{empty}

%%%%%%%%% ABSTRACT
\begin{abstract}
blablabla
\end{abstract}

%%%%%%%%% BODY TEXT
\section{Introduction}

blablabla

\section{Datasets}

blablabla

\subsubsection{Data preprocessing}

blablabla

\section{Models involved}

\section{Metrics}

blablabla

\subsection{Results}

\section{Testing}

\section{Conclusions}
blablabla

%\subsection{Suggested Structure}

%The following is a suggested structure for your report:

%\begin{itemize}
%	\item Introduction (20\%): describe the problem you are working on, why it's important, what are your goals, and provide also an overview of your main results.
%	\item Dataset (20\%): describe the data you are working with for your project. What type of data is it? Where did it come from? How much data are you working with? Did you have to do any preprocessing, filtering, etc., and why?
%	\item Method (30\%): discuss your approach for solving the problems that you set up in the introduction. Why is your approach the right thing to do? Did you consider alternative approaches? It may be helpful to include figures, diagrams, or tables to describe your method or compare it with others.
%	\item Experiments (30\%): discuss the experiments that you performed. The exact experiments will vary depending on the project, but you might compare with prior work, perform an ablation study to determine the impact of various components of your system, experiment with different hyperparameters or architectural choices. You should include graphs, tables, or other figures to illustrate your experimental results.
%\end{itemize}

%-------------------------------------------------------------------------
{\small
\bibliographystyle{ieee_fullname}
\bibliography{egbib}
}

\end{document}