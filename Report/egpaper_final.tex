\documentclass[10pt,twocolumn,letterpaper]{article}

\usepackage{cvpr}
\usepackage{times}
\usepackage{epsfig}
\usepackage{graphicx}
\usepackage{amsmath}
\usepackage{amssymb}
\usepackage{listings}
\usepackage{xcolor}
\usepackage[table,xcdraw]{xcolor}
\usepackage{caption}
\usepackage{makecell}
\usepackage[T1]{fontenc}
\usepackage{placeins}
\usepackage{booktabs}

\definecolor{lightergray}{rgb}{0.9,0.9,0.9}
\renewcommand{\arraystretch}{1.15}
\captionsetup[table]{skip=5pt}

\lstset{
    language=Python,
    basicstyle=\ttfamily\small,
    keywordstyle=\color{blue},
    stringstyle=\color{red},
    commentstyle=\color{green},
    showstringspaces=false,
    frame=single,
    breaklines=true
}
% Include other packages here, before hyperref.

% If you comment hyperref and then uncomment it, you should delete
% egpaper.aux before re-running latex.  (Or just hit 'q' on the first latex
% run, let it finish, and you should be clear).
\usepackage[breaklinks=true,bookmarks=false]{hyperref}

\cvprfinalcopy % *** Uncomment this line for the final submission

\def\cvprPaperID{****} % *** Enter the CVPR Paper ID here
\def\httilde{\mbox{\tt\raisebox{-.5ex}{\symbol{126}}}}

% Pages are numbered in submission mode, and unnumbered in camera-ready
%\ifcvprfinal\pagestyle{empty}\fi
\setcounter{page}{1}
\begin{document}

%%%%%%%%% TITLE
\title{A Neural Network Approach to Classification of Protein Residue Interactions}

\author{Andrea Auletta\\
{\tt\small andrea.@studenti.unipd.it}
% For a paper whose authors are all at the same institution,
% omit the following lines up until the closing ``}''.
% Additional authors and addresses can be added with ``\and'',
% just like the second author.
% To save space, use either the email address or home page, not both
\and
Marco Bernardi\\
{\tt\small marco.bernardi.11@studenti.unipd.it}
\and
Niccolò Zennaro\\
{\tt\small niccolo.zennaro@studenti.unipd.it}
}

\maketitle
%\thispagestyle{empty}

\begin{abstract}
    Bhoooooooooooooo
\end{abstract}

%%%%%%%%% BODY TEXT
\section{Introduction}

Residue Interaction Networks (RINs) are a representation of the non-covalent interactions between amino acid residues within a protein structure, derived based on their geometrical and physico-chemical properties. These networks provide a detailed mapping of intra-protein contacts, which are crucial for understanding the structural and functional dynamics of proteins. 
RING (\url{https://ring.biocomputingup.it/}) is a computational tool that facilitates the analysis of these networks by processing Protein Data Bank (PDB) files (\url{https://www.rcsb.org/}) to identify and classify residue-residue interactions within a given protein structure. 
RING categorizes these interactions into distinct contact types, including Hydrogen Bonds (HBOND), Van der Waals interactions (VDW), Disulfide Bridges (SBOND), Salt Bridges (IONIC), $\pi$-$\pi$ Stacking (PIPISTACK), $\pi$-Cation Interactions (PICATION), Hydrogen-Halogen Interactions (HALOGEN), Metal Ion Coordination (METAL\_ION), $\pi$-Hydrogen Bonds (PIHBOND), and a category for Unclassified Contacts.

This project is centered around the development of a predictive model that can infer the RING classification of residue contacts using statistical or supervised learning approaches, as opposed to purely geometrical methods. The objective is to design a program that calculates the likelihood or propensity of a residue-residue interaction belonging to each contact type defined by RING, starting from the structural data of the protein.


\section{Data Source}

The dataset utilized in this study comprises a collection of training examples derived from 3,299 Protein Data Bank (PDB) structures. Each PDB structure is represented by a separate file, which provides detailed information on the residue-residue contacts identified within the corresponding protein. The files are available for download and are organized such that each file contains a tab-separated table of interactions for a single PDB structure. 

Across all 3,299 PDB structures, the dataset contains a total of 2,476,056 residue-residue contacts, distributed among various interaction types as follows:

\begin{table}[h!]
\centering
\begin{tabular}{|l|r|}
\hline
\textbf{Contact Type} & \textbf{Count} \\
\hline
HBOND & 901,814 \\
VDW & 640,469 \\
PIPISTACK & 32,965 \\
IONIC & 30,355 \\
SSBOND & 1,792 \\
PICATION & 7,623 \\
PIHBOND & 1,836 \\
Unclassified & 860,202 \\
\hline
\end{tabular}
\caption{Distribution of contact types across the dataset.}
\end{table}

Each file is formatted as a tab-separated table, consisting of columns that provide essential details for each contact. The columns include residue identifiers, which follow the same naming conventions as those used in BioPython, along with several pre-calculated features. The final column in each table specifies the type of interaction, categorizing the contact according to the RING-defined classifications.

\subsection{Data Preprocessing}

The files containing the information for each PDB structure were merged into a single dataframe, creating the dataset for the model. This dataset underwent a preprocessing phase, which involved several crucial steps for data cleaning and preparation.

First, the null values present in the dataset were replaced with the mean of the corresponding column values. This operation ensured data continuity, minimizing the impact of missing values on the model's performance.

Furthermore, all rows lacking a classification, i.e., without a value in the "interaction" column, were reclassified under the "Unclassified" category. This update involved modifying the values in the "interaction" column, ensuring that every interaction in the dataset was consistently labeled according to the contact categories defined by RING.


\section{Models involved}


\section{Fine tuning}


\section{Testing}


\section{Conclusions}

{\small
\bibliographystyle{ieee_fullname}
\bibliography{egbib}
}

\end{document}